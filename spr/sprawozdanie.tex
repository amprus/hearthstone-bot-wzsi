\documentclass[a4paper, 12pt]{article}

\usepackage{polski}
\usepackage[utf8]{inputenc}
\usepackage{a4wide}
\usepackage[OT4]{fontenc}
\usepackage[english,polish]{babel}
\usepackage{indentfirst}
\usepackage{graphicx} 
\usepackage{float}
\usepackage{listings}
\usepackage[hidelinks, pdftoolbar = false, pdfmenubar = false, bookmarks = false]{hyperref}
\usepackage{geometry}
    
\renewcommand{\baselinestretch}{1.0}
\newgeometry{tmargin=2.5cm, bmargin=2.5cm, lmargin=2.5cm, rmargin=2.5cm} 
    
\begin{document}

\thispagestyle{empty}
\noindent Aleksander Prus, 218404 \hfill Wrocław, d.\ \today
\\ Rafał Szyszka, 218207

\vfill

\begin{center}
    \begin{Huge}
        Monte Carlo Tree Search\\
    \end{Huge}
\end{center}

\begin{center}
    Rok akad. 2017/2018, DAN
\end{center}

\vspace{0.2ex}

\begin{flushright}
    \begin{minipage}[t]{0.4\columnwidth}
        \noindent Mgr inż. Jan Jakubik
    \end{minipage}
\end{flushright}

\vfill
\newpage
\tableofcontents
%%%%%%%%%%%%%%%%%%%%%%%%%%%%%%%%%%%%%%%%%%%%%%%%%%%%%%%%%%%%%%%%%%%%%%%%%%%%%%%%%%%%%%%%%%%%%%%%%%%%%%%%%%%%%%%%%%%%%%%%%%%%%%%%%%%%%%%%%%%%%%%%%%%%%%%%%%%%
\newpage
\section*{Cel ćwiczenia}
Celem ćwiczenia była iplementacja uproszczonej wersji popularnej gry Hearthstone, z graczem komputerowym, którego zachowanie zdefiniowane będzie algorytmem Monte Carlo Tree Search. Dodatkowo należało zaimplementować trzech graczy naiwnych:
\begin{itemize}
\item losowy,
\item agresywny,
\item kontrolujący.
\end{itemize}

Gracz agresywny w pierwszej kolejności skupia się na zadaniu jak największej ilości obrażeń bohaterowi gracza. Natomiast gracz kontrolujący skupia się na wyeliminowaniu minionów przeciwnika.


\section*{Realizacja}
Do realizacji zadania wybrano język Python, ze względu na mnogość zewnętrznych bibliotek oraz wysoki poziom abstrakcji języka.

\subsection*{Naiwni gracze}
Logika naiwnych graczy została zimplementowana w pakiecie \textbf{naive-bots}. Ich zachowanie określają dwie metody \textbf{card-rating} oraz \textbf{choose-targets}. Pierwsza z nich odpowiada za dobór kart do zagrania z ręki, w turze wybrangeo bota. Natomiast druga metoda, odpowiada za wybór celów do zaatakowania po stronie przeciwnika.

\subsubsection*{Bot losowy}
Ten gracz w sposób losowy wybiera karty do zagrania oraz cele, które zaatakuje w danej turze.

\subsubsection*{Bot agresywny}
Ten gracz preferuje karty z dużą ilością ataku. Za swoje cele, w pierwszej kolejności, bierze karty przeciwnika z właściwością \textbf{[TAUNT]}, następnie wybiera karty z wysoką wartością życia (większy od 4), jeżeli takich kart przeciwnik nie posiada na stole, bot agresywny atakuje jego bohatera.

\subsubsection*{Bot kontrolujący}
Ten gracz preferuje karty defensywne, z dużą ilością życia. Za swoje cele, w pierwszej kolejności, bierze karty przeciwnika z właściwością \textbf{[TAUNT]}, następnie wybiera karty z wysoką wartością życia (większy od 4), następnie wybiera pozostałe karty przeciwnika na stole. Bohater przeciwnika jest ostatnim wybieranym celem gracza kontrolującego.

\end{document}